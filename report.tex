\documentclass{article}

\usepackage{enumerate}
\usepackage[bottom=1in,top=1in]{geometry}
\usepackage{parskip}
\usepackage{graphicx}
\usepackage{subcaption}
\usepackage{multirow}
\usepackage{amssymb}
\usepackage[pdftex,colorlinks,urlcolor=blue]{hyperref}

\geometry{letterpaper}

\begin{document}

\title{CS276 PA4 Report}

\author{
  Jiawei Yao\\
  \texttt{jwyao@stanford.edu}
  \and
  Wei Wei\\
  \texttt{wwei2@stanford.edu}
}

\maketitle

\section{Task 1 - Pointwise Approach with Linear Regression}

In pointwise approach, we use \texttt{LinearRegression} model with TF-IDF features.
We apply the following tweaks to TF-IDF features:

\begin{itemize}
    \item add-one smoothing on IDF
    \item sublinear on raw term frequencies
    \item body length normalization with added length of 900
\end{itemize}

After the model has been trained, for documents of given query, we use \texttt{Collections.sort} on the documents with a comparator such that
for two documents $d_1,d_2$, the one with higher score from the trained LR model is ranked higher.

NDCG on train and dev datasets:

\begin{table}[!htb]
    \centering
    \begin{tabular}{| r | l | l |}
        \hline
        & \textbf{Train} & \textbf {Dev.} \\
        \hline
        \textbf{LR} & 0.8705 & 0.8420 \\
        \hline
    \end{tabular}
    \caption{NDCG on Task 1}
\end{table}

\section{Task 2 - Pairwise Approach and Ranking SVM}

For this task, we use the same TF-IDF features as in Task 1 with the same tweaks.

In training SVM, the following techniques are used:

\begin{itemize}
    \item use \texttt{Standardize} to standardize document features (not the differences)
    \item generate evenly distribtued features for each label by computing differences of document features
\end{itemize}

\texttt{Collections.sort} is used to rank documents of a same query.
For documents $d_1,d_2$, the comparator

\begin{enumerate}
    \item standardizes the document features with \emph{the same} \texttt{Standardize} from training
    \item computes the difference
    \item feeds the difference vector to the trained SVM
        \begin{itemize}
            \item if the label output is positive, $d_1$ is ranked higher
            \item otherwise $d_1$ is considered as \emph{not higher than} $d_2$\footnote{In this case, we let the sort algorithm to break tie.}
        \end{itemize}
\end{enumerate}

For Linear SVM, we used default parameters. For RBF SVM, we used $C=8.0,\gamma=0.001$.

NDCG on train and dev datasets:

\begin{table}[!htb]
    \centering
    \begin{tabular}{| r | l | l |}
        \hline
        & \textbf{Train} & \textbf {Dev.} \\
        \hline
        \textbf{Linear SVM} & 0.8653 & 0.8463 \\
        \hline
        \textbf{RBF SVM} & 0.8660 & 0.8458 \\
        \hline
    \end{tabular}
    \caption{NDCG on Task 2}
\end{table}

\section{Task 3 - More Features and Error Analysis}

\begin{table}[!htb]
    \centering
    \begin{tabular}{| r | l | l |}
        \hline
        & \textbf{Train} & \textbf {Dev.} \\
        \hline
        \textbf{RBF} & \textbf{TODO} & 0.8619 \\
        \hline
    \end{tabular}
    \caption{NDCG on Task 3}
\end{table}

\subsection{Implementation Decisions}

\begin{itemize}
    \item Use BM25 with pagerank. BM25 weights are copied from PA3. Use BM25 without PageRank give as lower score $85.23\%$ compared to $85.56\%$ using PageRank.
    \item Adjusting $K1$ in BM25 doesn't give us significant increase in NDCG score. We tried 1, 1.5, 2, 2.5, and 3, they all give us similar scores. The highest is $85.60\%$ while the lowest is $85.56\%$.
    \item Use raw PageRank or log(PageRank). The NDCG results are similar so we use raw PageRank for simplicity.
\end{itemize}

\subsection{Combination of suggested features}

We experiment with three suggested features first. As we can see bm25 is the most important feature. Only using bm25 can boost performance by $1\%$. Smallest window is not a very appealing feature. (In PA3 smallest window also provides marginal performance boost.)

As Non-linear SVM requires grid search, we only use linear SVM to report NDCG scores.

\begin{table}[!htb]
  \centering
  \begin{tabular}{|c|c|c|c|c|c|c|}
    \hline
    \multirow{2}{*}{bm25} & \multirow{2}{*}{window} & \multirow{2}{*}{pagerank} & \multicolumn{2}{c|}{Linear SVM} & \multicolumn{2}{c|}{Non-linear SVM} \\ \cline{4-7}
                          & & & train & dev & train & dev \\ \cline{4-7}
    \hline
    \checkmark & & & 87.16 & 85.57 & - & - \\
    \hline
    & \checkmark & & 86.53 & 84.62 & - & - \\
    \hline
    & & \checkmark & 87.36 & 85.22 & - & - \\
    \hline
    \checkmark & \checkmark & & 87.16 & 85.53 & - & - \\
    \hline
    & \checkmark & \checkmark & 87.37 & 85.16 & - & - \\
    \hline
    \checkmark & & \checkmark & 87.06 & 85.56 & - & - \\
    \hline
    \checkmark & \checkmark & \checkmark & 87.06 & 85.56 & 86.55 & 84.43\\
    \hline
  \end{tabular}
  \caption{Results for task 3 - basic requirement}
\end{table}

\subsection{Systematic Errors}

After experimenting with three suggested features, we found three types of systematic errors.

\begin{enumerate}
  \item Our system favors long urls that have higher term count.
    \begin{itemize}
      \item For query "mscs program sheet", our system ranks \url{http://cs.stanford.edu/degrees/mscs/programsheets/09-10/MSCS-0910-RWC.pdf} the highest and \url{http://cs.stanford.edu/degrees/mscs/programsheets/} only ranks the 4th. The first url is more specific and contains information about RWC track and it is for 09-10 academic years. The first link is too specific while the second one is just right. Hence, we take url length into consideration by adding a new numeric feature.
    \end{itemize}
  \item Our system gives ".pdf" file pretty low score. Thus we add a binary feature to indicate whether a document name ends with ".pdf".
  \item Our system doesn't take current time into consideration. If user searches for 2014 academic calendar, we should give 2012-2013 academic calendar a high score. We don't add this one as a new feature and it would potentially raise our NDCG score.
\end{enumerate}

\subsection{More features}

We took three additional features into consideration:

\begin{enumerate}
    \item Number of fields that query words appear in.
    \item Url length.
    \item Body length.
\end{enumerate}

\textbf{Number of fields that query words appear in} is a extremely useful feature and boost our performance a lot. Basically, it assumes that if all five fields contains query words, the document may be better.

\textbf{Url length} makes sense because shorter url tends to be better based on our observation. For example, two urls are \url{http://nlp.stanford.edu/manning/tex/} and \url{http://nlp.stanford.edu/manning/tex/avm.sty} and the query is christopher manning latex macros, the former urls is shorter than the latter while the former contains all the key word of the latter. The latter may be more specific and contains less information of our information need.

Considering \textbf{Body length} is reasonable because if one document has long body length but has the same body hits as a shorter document, the shorter document tends to be more relevant than the longer document.

However, based on our experiment, not all three features are useful.

\section{Extra Credit}

For extra credit, we use SVM regression feature provided for the pointwise approach.
Specifically, we use NuSVR SVM with RBF kernel, $C=16.0,\nu=0.25$. We also tried standardizing data but that was not as good as expected.

NDCG on train and dev datasets (numbers in parentheses are improvement over Task 1):

\begin{table}[!htb]
    \centering
    \begin{tabular}{| r | l | l |}
        \hline
        & \textbf{Train} & \textbf {Dev.} \\
        \hline
        \textbf{NuSVR} & 0.8669 (-0.42\%) & 0.8483 (0.75\%) \\
        \hline
        \textbf{NuSVR w/ Std.} & 0.8770 (0.74\%) & 0.8328 (-1.09\%) \\
        \hline
    \end{tabular}
    \caption{NDCG on Extra Credit}
\end{table}

We can see that NuSVR generalizes better than linear regression - NDCG performance is slightly lower on train set but higher on dev set.

\end{document}
